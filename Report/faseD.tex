%---------------------------------------------------------------------------------------------------------------%
\section{Fase --- D}
%---------------------------------------------------------------------------------------------------------------%

Nesta fase, requere-se que se implemente a funcionalidade de refrescamento da
tabela de números hexadecimais, em intervalos de $\frac{1}{R}$ segundos, onde
\texttt{R} é o rácio de refrescamento em \emph{Hertz}.  Para o refrescamento,
considere-se $M_{T,K}$, onde $N=T$, $D=K$, e a matriz é circular, ou seja cada
coluna e cada linha é uma \emph{buffer} circular. Em consequência devem ser
aplicados os seguintes passos:

\begin{enumerate}
	\item $X$ é somatório do valor binário de todos os símbolos de chave de
		da operação de \emph{reset} (cada símbolo tem oito bits);

	\item$S_1$ é somatório do valor decimal de todos os dígitos $d_{i,j}$ de $M$ em
		que $i$ e $j$ são ambos pares;

	\item$S_2$ é somatório do valor decimal de todos os dígitos $d_{i,j}$ de $M$
		em que $i$ e $j$ são ambos ímpares;

	\item$C$ é igual ao resto da divisão inteira de $(S_1 + X)$ por
		$D$;\footnote{\emph{No enunciado $D$ estava trocado por $N$}}
                                                               
	\item$L$ é igual ao resto da divisão inteira de $(S_2 + X)$ por
		$N$;\footnote{\emph{ibidem}}

	\item Aplicar o refrescamento vertical   $v(C+1,1,M)$;

	\item Aplicar o refrescamento horizontal $h(L+1,1,M)$;

	\item Aplicar a operação de substituição $s(L+1,C+2,M)$;\footnote{\emph{No
		enunciado estava $s(L,C+2,M)$. No entanto, de acordo com a definição $L$ tem
	que ser maior ou igual a 1}}
\end{enumerate}


\newpage

As definições das operações $v$, $h$ e $s$ figuram a seguir:

\begin{itemize}
	\item$v(i, j, M_{T,K}$) 
	\item$h(i, j, M_{T,K}$)  
	\item $s(i, j, M_{T,K}$)
		\begin{itemize}
  \item Substituição da linha $i$ pelo resultado do \texttt{XOR} da linha: 
		\begin{itemize}
			\item $\begin{cases}
					\text{Se } i = 1~\text{Então}~i - 1 = T\\
					      i -1~\text{Caso contrário}
				     \end{cases}$ 
			\item $\begin{cases}
					\text{Se } i = T~\text{Então}~i + 1 = 1\\
					      i + 1~\text{Caso contrário}
				     \end{cases}$ 
		\end{itemize}
	\item  Substituição da coluna $j$ pelo resultado do \texttt{XOR} da coluna: 
		\begin{itemize}
			\item $\begin{cases}
					\text{Se } j = 1~\text{Então}~j - 1 = K\\
					      j - 1~\text{Caso contrário}
				     \end{cases}$ 
			\item $\begin{cases}
					\text{Se } j = K~\text{Então}~j + 1 = 1\\
					      j + 1~\text{Caso contrário}
				     \end{cases}$ 
		\end{itemize}
		\end{itemize}
\end{itemize}


As funcionalidades do \emph{refreshing} da tabela de números hexadecimais são
externas aos objetos da MIB, na medida que pertencem à classe
\emph{AgentHelper}, mas são partilhadas por processos que escrevem e leem dos
objetos, neste caso, a tabela de números hexadecimais. Assim, decidiu-se
adicionar 5 métodos e uma variável de instância na classe mencionada,i\.e,
os métodos \texttt{horizontal}, \texttt{vertical}, \texttt{substitution},
\texttt{refresh} e o tempo de refrescamento.

O tempo de refrescamento é calculado, logo na instanciação do
\emph{AgentHelper}, sendo mantido durante toda a execução do agente. O método
\texttt{refresh} aplica os cálculos já descritos acima, bem como a aplicação do
dos métodos \texttt{horizontal}, \texttt{vertical}, \texttt{substitution},
baseados em $v(i, j, M_{T,K}$), $h(i, j, M_{T,K}$), $s(i, j, M_{T,K}$),
respetivamente. Note-se que, se teve que se proceder à conversão de índices base
1, para índices base 0. Os valores da tabela da MIB são obtidos através de um
iterador e colocados num \emph{array} de \emph{strings} temporário, e das
manipulações numa matriz, obtida para o efeito da tabela, colocado de novo num
\emph{array} de \emph{strings} e com outro iterador alteram-se os valores com
os valores do último \emph{array}. 

\newpage

A \emph{Figura~\ref{fig:fasec:calcx}} representa o trecho de código onde foi
implementado o cálculo de $X$. Cada caractere em JAVA corresponde a um
\emph{byte}, ou seja, 8 bits, podendo somar-se diretamente ao estilo da
linguagem de programação C. 

\begin{center}
\begin{minted}{java}
for (int i = 0; i < this.getResetKey().length(); i++) {
	X += this.getResetKey().charAt(i);
}
\end{minted}
 	\captionsetup{type=figure, width=0.8\linewidth}
	\caption{Cálculo de $X$}
\label{fig:fasec:calcx} 
\end{center}


A \emph{Figura~\ref{fig:fasec:s1s2}} corresponde ao cálculo de $S_1$ e $S_2$,
vem como constrói uma matriz de inteiros, entre 0 e 15 (valores decimais
possíveis de um dígito hexadecimal), para as operações de deslocamento
e substituição.

\begin{center}
\begin{minted}{java}
for (int i = 0; i < N; i++) {
	for (int j = 0; j < D; j++) {

		m[i][j] = Integer.parseInt(seed[i].charAt(j) + "", 16);
		if (((i % 2) == 0) && ((j % 2) == 0)) {
			S1 += Integer.parseInt(seed[i].charAt(j) + "", 16);
		} else if (((i % 2) != 0) && ((j % 2) != 0)) {
			S2 += Integer.parseInt(seed[i].charAt(j) + "", 16);
		}

	}

}

\end{minted}
 	\captionsetup{type=figure, width=0.8\linewidth}
	\caption{Cálculo de $S_1$, $S_2$ e construção de matriz de inteiros}
\label{fig:fasec:s1s2} 
\end{center}

O cálculo de $C$ e $L$ figura abaixo.

\begin{center}
\begin{minted}{java}
int C = (S1 + X) % D;
int L = (S2 + X) % N;
\end{minted}
 	\captionsetup{type=figure, width=0.8\linewidth}
	\caption{Cálculo de $C$ e $L$}
\label{fig:fasec:} 
\end{center}


A \emph{Figura~\ref{fig:fasec:ops}} representa a aplicação das operações. 

\begin{center}
\begin{minted}{java}
vertical(C + 1, 1, N, m);
horizontal(L + 1, 1, D, m);
substitution(L + 1, C + 2, N, D, m);
\end{minted}
 	\captionsetup{type=figure, width=0.8\linewidth}
	\caption{Aplicação de \texttt{vertical}, \texttt{horizontal}
	e \texttt{substitution}}
\label{fig:fasec:ops} 
\end{center}

No método \texttt{main}, no ciclo infinito, aplica-se a operação de
\emph{refresh}, sendo o processo pausado durante o tempo de refrescamento
previamente calculado.
